\documentclass{ecnreport}

\stud{Control \& Robotics Master}
\topic{Manipulator Modeling \& Control}

\begin{document}

\inserttitle{Manipulator Modeling \& Control}

\section{Content of this lab}

The goal of this lab is to program in C++ the three fundamentals models for robot manipulators:
\begin{itemize}
	\item Direct Geometric Model: what is the pose of the end-effector for a given joint configuration?
	\item Inverse Geometric Model: what joint values correspond to a given end-effector pose?
	\item Kinematic Model: what is the velocity screw of the end-effector when the joints move?
\end{itemize}
These models will be applied on three different robots, and used to perform basic point-to-point control.\\

As it is not a lab on C++, most of the code is already written and you only have to fill the required functions:
\begin{itemize}
	\item \texttt{Robot::fMw(q)} for the direct geometric model wrist-to-fixed frame
	\item \texttt{Robot::inverseGeometry(M)} for...well, the inverse geometry
	\item \texttt{Robot::fJw} for the kinematic model wrist-to-fixed frame
\end{itemize}

This project uses the ROS\footnote{Robot Operating System, http://www.ros.org} framework which imposes some particular steps to configure the environment. They are detailed in Appendix \ref{ros}.

The {\bf only} files to be modified are:
\begin{itemize}
	\item \texttt{control.cpp}: main file where the control is done depending on the current robot mode
	\item \texttt{robot\_turret.cpp}: model of the RRP turret robot
	\item \texttt{robot\_kr16.cpp}: model of the industrial Kuka KR16 robot
	\item \texttt{robot\_ur10.cpp}: model of the Universal Robot UR-10
\end{itemize}
We will use the ViSP\footnote{Visual Servoing Platform, http://visp.inria.fr} library to manipulate mathematical vectors and matrices, including frame transforms, rotations, etc. The main classes are detailed in Appendix \ref{visp}.\\

\section{The robots}

Three robots are proposed with increasing complexity. It is strongly advised to start with the Turret, then the Kuka KR16 then the UR-10.\\
For each of these robots, the table of modified Denavit-Hartenberg parameters should be defined.

\section{Building the models}




\appendix


\section{How to load the C++ project inside Qt Creator}\label{ros}

An actual ROS course will be held in the second semester, for now just configure as follows.\\
You should have a folder named \texttt{ros} in your home directory. Open a terminal inside and follow these steps:
\begin{enumerate}
	\item Clone this project inside  \texttt{ros/src}
	\begin{center}\cppstyle
		\begin{lstlisting}
		mkdir src
		cd src
		git clone https://github.com/oKermorgant/ecn_manip.git
		\end{lstlisting}
	\end{center}
	\item Compile using catkin (you'll discover soon enough what it is) : \texttt{catkin build}
	\item Run Qt Creator from the top icon
	\item Load the \texttt{ros/src/ecn\_manip/CMakeLists.txt} file through \texttt{File...open project}
	\item QtCreator asks for a compilation folder: give \texttt{ros/build/ecn\_manip}
	\item The files should be displayed and ready to compile and run
	\item Compilation is done by clicking the bottom-left hammer
	\item Run your program with the green triangle. It can be stopped by clicking on the red square
	
\end{enumerate}

\section{Using ViSP}\label{visp}

\end{document}
